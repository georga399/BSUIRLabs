\sectioncentered*{Введение}
\addcontentsline{toc}{section}{Введение}
\label{course:introduction}

Серверная часть приложения аукциона - это важная часть системы, 
которая обеспечивает автоматизацию основных процессов аукциона. 
Она позволяет упростить и оптимизировать процессы хранения, создания,
поиска товаров аукциона, а также управление информацией о пользователях и их ставках. 

Целью данного курсового проекта является создание и разработка серверной части приложения аукциона, 
которая позволит автоматизировать основные процессы аукциона. 
Это включает в себя упрощение и оптимизацию процессов хранения, поиска, создания аукционов, создания ставок. 

Задачи данного курсового проекта:

– провести анализ требований к приложению аукциона, определить функциональные требования;
   
– разработать архитектуру приложения, определить ее основные компоненты и интерфейс;

– разработать и реализовать программное обеспечение;

– сформулировать выводы, исходя из нашей работы.

В данном курсовом проекте будут рассмотрены такие главы, 
как анализ предметной области, 
проектирование программного средства, 
разработка программного средства, 
проверка работоспособности приложения и руководство пользователя.